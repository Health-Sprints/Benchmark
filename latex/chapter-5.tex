\documentclass{article}

\usepackage{hyperref}
\begin{document}

\title{CRI004: Communications}

\maketitle


Good communications are a key component of successful crisis management. Here we need to distinguish between (1) 'internal' communications within the administrative/crisis team, which is governed by the official regulations of the team, and (2) 'external' communications. Internal communications are covered in the chapter 'Operational planning'. This chapter describes 'external' communications with the public.


Here we draw a distinction between 'risk communication' and 'crisis communication'. 'Risk communication' refers to communication before a crisis arises. 'Crisis communication' becomes necessary once the crisis occurs. Good risk communication makes it considerably easier to communicate during the crisis, because it creates knowledge on which crisis management can build.


The \textbf{\href{https://www.bmi.bund.de/SharedDocs/downloads/DE/publikationen/themen/bevoelkerungsschutz/leitfaden-krisenkommunikation.pdf}{Guidelines on crisis communication}} published (in German only) by Germany's Federal Ministry of the Interior, Building and Community (BMI) provides a clear overview of the following topics for effective communication:

\begin{itemize}
\item Risk communication


\item Crisis communication


\item Target-group appropriate crisis communication


\item Crisis communication plan


\item Planning aids


\end{itemize}

Good communication can enable the public, media representatives and public authorities to become an effective team for managing a crisis.


\subsection{Risk communication}\label{H9313720}



The aim of risk communication is to build up the mutual confidence and trust of all stakeholders. This is best achieved by forming long-term relationships. These are the basis for credibility, which is essential in a crisis. Risk communication therefore relies on transparency, reliability and maximum honesty. Accordingly, risk communication is a continuous process. Public authorities should therefore look upon it and use it as a

\begin{quote}



'sharing of information and opinions on risks, risk prevention, risk minimisation and risk acceptance' (Federal Office for Civil Protection and Disaster Assistance 2011),


\end{quote}


that involves all stakeholders.


This particularly important when the public are to be motivated to themselves perform risk management. One example of crisis preparedness is the brochure published by the BBK entitled \textbf{\href{https://www.bbk.bund.de/SharedDocs/Downloads/BBK/DE/Publikationen/Broschueren_Flyer/Buergerinformationen_A4/Ratgeber_Brosch.pdf}{Guidance on emergency preparedness}}\href{https://www.bbk.bund.de/SharedDocs/Downloads/BBK/DE/Publikationen/Broschueren_Flyer/Buergerinformationen_A4/Ratgeber_Brosch.pdf}{ }\textbf{\href{https://www.bbk.bund.de/SharedDocs/Downloads/BBK/DE/Publikationen/Broschueren_Flyer/Buergerinformationen_A4/Ratgeber_Brosch.pdf}{and what to do in an emergency}} (German only), which contains important information and checklists

\begin{quote}



'on all key topics – from food supplies to the emergency bag – for personal emergency preparedness' (Federal Office for Civil Protection and Disaster Assistance, 2018).


\end{quote}


Only preventive communication will enable a crisis response that is able to draw on the preparedness and knowledge in place without delay.


One example of the need for risk communication is the shortage of resources described in this manual, which in this instance involves the distribution of medicines. As long as there is no risk to their health, any individual will understand and accept that medical personnel who have to take care of the first patients must be protected first. So must the personnel responsible for maintaining critical infrastructures or public safety. For work within your own office, this can mean establishing before a crisis occurs which tasks that are otherwise an important part of routines can be deprioritised during the crisis.


One of the challenges for risk communication is that the perception of risk is influenced by many factors. This can mean that major risks are underestimated and minor risks overestimated, with the result that the measures undertaken by public health offices are seen as either excessive or insufficient, Here is an example: A vaccination is refused because of the fear of extremely rare side effects, and the much higher risk of severe illness through infection is implicitly accepted. If dangerous diseases occur only rarely due to high vaccination rates, separating the risk-benefit calculations for the individual and for society becomes particularly challenging.


How a particular risk is communicated also plays a crucial role in how it is perceived. For example, various studies indicate that with respect to the likelihood of occurrence, relative figures often lead to a higher assessment of the risk than absolute figures (Wegwarth, Odette; Gigerenzer, Gerd 2011). How a message is put across can also lead to a misjudgement of the damage that will be done. This plays an especially major role with biological emergencies, in which the extent of damage depends on so many factors that it is almost impossible to assess. This uncertainty must also be communicated. For example, strategic plans for dealing with a 30\% absence of personnel have led to the firm belief that roughly 30\% of people fall ill in any pandemic.


The timing of risk communication can also affect the perception of risk. Publication of such emergency-related information therefore needs to be placed in the context of a risk assessment which makes clear to the target audience whether the information is of a general nature and of no current relevance, or whether it is designed to prepare recipients for an incident expected in the near future. As well as building long-term trust and confidence, risk communication can also be used in the short term in case of assumed, foreseeable risks (e.g. extreme weather events) to

\begin{quote}



'raise public awareness of forthcoming events and prepare measures to warn and protect the public,


\end{quote}


as the BBK writes in its glossary \textbf{\href{https://www.bbk.bund.de/SharedDocs/Downloads/BBK/DE/Publikationen/Praxis_Bevoelkerungsschutz/Glossar_2018.pdf}{Selected key terms in}}\href{https://www.bbk.bund.de/SharedDocs/Downloads/BBK/DE/Publikationen/Praxis_Bevoelkerungsschutz/Glossar_2018.pdf}{ }\textbf{\href{https://www.bbk.bund.de/SharedDocs/Downloads/BBK/DE/Publikationen/Praxis_Bevoelkerungsschutz/Glossar_2018.pdf}{public protection}}' (German only).


\subsection{Crisis communication}\label{H2846710}



Unlike risk communication, which focuses on crisis preparedness, crisis communication involves

\begin{quote}



'sharing information and opinions during a crisis in order to prevent or limit damage to a protected good' (Federal Office for Civil Protection and Disaster Assistance, 2011).


\end{quote}


Crisis communication must ensure that all responsible persons/bodies have the same level of information and knowledge. Equally, the media and the general public must be informed as truthfully, transparently and promptly as possible.


Crisis communication aims to provide a sufficient amount of specific information in good time, so that the measures needed to protect the population can be taken. To avoid uncertainty an agreed language regime is required, which all stakeholders must adhere to. This is why crisis communication also needs to be planned in advance.


If communication is poorly timed or if conflicting information is transmitted, there is a risk of excessive misinformation and false rumours that in the worst case may lead to hysteria or panic. The initial reactions may already be crucial in determining whether the organisational structures lose control of how the crisis develops.


\textbf{N.B.: To maintain control over communication, please therefore apply the following principles:}

\begin{enumerate}
\item \textbf{Act, don't react}


\item \textbf{Release only reliable information, and explain any areas of uncertainty}


\item \textbf{Maintain contact and be accessible}


\end{enumerate}

\subsection{Press and media work}\label{H3071540}



In a crisis, external communication must be clearly regulated. The press and the media need permanent contact persons. Information may only be communicated to the public by authorised persons who are specially trained to perform this task. People making enquiries must be referred to authorised communicators. Specialist expertise does not automatically entail an ability to communicate it comprehensibly. In public administrations, press work is usually managed through a press office. This press office must be called in.


Since the public expect the authorities to provide timely assistance, give assurances and function efficiently, dealing with uncertainties – which often exist when a crisis begins – is a particular challenge. To maintain trust and confidence, this uncertainty must be communicated. At the same time the authorities must identify the measures that will be taken to clarify the uncertainties and say when they expect further information to be available.


When preparing for a press conference it is important not only to prepare the information that the authorities intend to communicate, but also to consider what information the press would like to have. If the media representatives do not receive answers to their questions from the authorised persons they will look for other 'experts' to provide answers. For highly complex and specialised topics it may therefore be advisable to invite a team of qualified individuals to the press conference. Depending on the scenario, this might include for instance the attending physician, emergency responders who are involved or representatives of other public authorities.


It is also helpful to provide the press representatives with a good organisational environment for their professional work. This includes for example a sufficiently large press room with good acoustics, if possible equipped with desks, chairs, plug points and wireless network, so that the press can process the information right away. If the time frame is longer it may be appropriate to provide catering, or at least drinks and possibly simple meals. If possible the press and media should be supported by a trained individual who is permanently present, or who at least can be reached at any time.


\subsection{Direct communication – Internet, hotline, social media \& apps}\label{H4511224}



As well as press conferences, which enable exclusively face-to-face contact between the authorities and media representatives, the Internet offers numerous possibilities for communicating with the public directly. This also requires advance preparation in order to be able to access the needed infrastructure and expertise when a crisis occurs. Knowledge of the target group is especially important for direct communication, to ensure that the right channels of communication and appropriate language are selected.


For online communication dark sites can be prepared in advance, which can then be fed with current information and quickly activated when a crisis occurs.


Another form of communication that is very popular with the public, but very personnel-intensive, are hotlines. Conversely, the most frequently asked questions also indicate the areas in which there is a particular need for information, or even where false information is being spread.


The Federal Emergency Information and News App, or \textbf{\href{https://www.bbk.bund.de/DE/NINA/Warn-App_NINA.html}{Warning App NINA}} for short, was specially developed to provide early warning of emergencies. Important messages warning of various hazards, such as the spread of hazardous substances or a major fire, are sent directly to the mobile phones of the public. Here too advance risk communication is required in order to publicise the app.


\subsection{Specialist communication}\label{H7417663}



In some crises it may be necessary or helpful to conduct targeted specialist communication with particular target groups. For the public health services, it would be particularly important to conduct such communication with doctors working in their area of responsibility. First of all, they too may be subject to uncertainty which urgently needs to be addressed. Secondly they are important multipliers who enjoy a high level of trust among the public.


Given their training, medical professionals can be expected to possess both medical knowledge and the language skills needed to communicate with the public.


\subsection{Responsibility and documentation}\label{H3455834}



Administrations usually have press offices with press spokespersons who are responsible for public relations. The handling of press enquiries from other authorities, enquiries from institutions and enquiries from the public should be clearly regulated. In crises, central information points are often set up that take on part of the public relations work. It is not advisable for experts to provide information in an unregulated manner. The persons authorised to provide information should be obliged to document enquiries in a transparent and structured way. The key persons responsible for public relations must be able at all times to ascertain what information has already been communicated.


Another thing that is important in communication, particularly in crises, is documentation, which must guarantee transparency. Clearly defined internal structures for responsibility and decision-making can facilitate timely and appropriate implementation via the communication channels. Here it is also essential to take into account the feedback from the implementing entities, as well as information from sections of the administration not involved and from the public.

\end{document}
